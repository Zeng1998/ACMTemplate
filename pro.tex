\documentclass[a4paper,12pt]{article}
\usepackage{zh_CN-Adobefonts_external} % Simplified Chinese Support using external fonts (./fonts/zh_CN-Adobe/)
\usepackage{fancyhdr}  % 页眉页脚
\usepackage{minted}    % 代码高亮
\usepackage[colorlinks,linkcolor=black,anchorcolor=black,citecolor=black]{hyperref}  % 目录可跳转
\setlength{\headheight}{15pt}
\usepackage{geometry}
\usepackage{framed} 
%\geometry{a4paper,scale=0.9}
\geometry{a4paper,left=2cm,right=2cm,top=2cm,bottom=2cm}
% 定义页眉页脚
\pagestyle{fancy}
\fancyhf{}
\fancyhead[C]{ACM Template By Zeng Xiaocan}
\lfoot{}
\cfoot{\thepage}
\rfoot{}

\author{Zeng Xiaocan}   
\title{ACM Template}

\begin{document} 
%\maketitle % 封面
%\newpage % 换页
%\tableofcontents % 目录
%\newpage
\section*{线段树}
\subsection*{多标记多次方}
\inputminted[]{c++}{code/hdu4578.cpp}
\subsection*{最长连续区间}
\inputminted[]{c++}{code/hdu1540.cpp}
\subsection*{多线段树可用连续区间}
\inputminted[]{c++}{code/hdu4553.cpp}
\subsection*{二分线段树查询}
\inputminted[]{c++}{code/hdu4614.cpp}
\subsection*{扫描线见part.pdf}
\subsection*{可持久化普通线段树}
\inputminted[]{c++}{code/hdu4348.cpp}
\subsection*{线段树维护区间状态转移矩阵}
\inputminted[]{c++}{code/cf705E.cpp}
\subsection*{区间加斐波那契数列}
\inputminted[]{c++}{code/cf446C.cpp}
\subsection*{区间加等差数列}
\inputminted[]{c++}{code/luoguP1438.cpp}
\section*{树套树}
\subsection*{带修区间第k小}
\inputminted[]{c++}{code/luoguP2617.cpp}
\subsection*{带修区间值域个数}
\inputminted[]{c++}{code/19ncI.cpp}
\subsection*{带插入区间第k大}
\inputminted[]{c++}{code/luoguP3332.cpp}
\section*{整体二分}
\subsection*{区间第k小}
\inputminted[]{c++}{code/luoguP3834.cpp}


%\section*{Graph Theory} % 一级标题
%\subsection*{Minimum Spanning Tree} % 二级标题
%\subsubsection*{Kruskal} % 三级标题
%\inputminted[breaklines]{c++}{graph/kruskal.cc} % 插入代码文件
%% 中文测试
%\subsection*{单源最短路}
%\subsubsection*{SPFA}
%\inputminted[breaklines]{c++}{graph/spfa.cc}

%\twocolumn  % 分页显示
%\newpage
%\section*{String}
%\subsection*{KMP}
%\inputminted[breaklines]{c++}{string/kmp.cc}

%\subsection*{Suffix Automaton}
%\inputminted[breaklines]{c++}{string/suffix-automaton.cc}

%\newpage
%\section*{Others}

\end{document}
