\documentclass[a4paper,12pt]{article}
\usepackage{zh_CN-Adobefonts_external} % Simplified Chinese Support using external fonts (./fonts/zh_CN-Adobe/)
\usepackage{fancyhdr}  % 页眉页脚
\usepackage{minted}    % 代码高亮
\usepackage[colorlinks,linkcolor=black,anchorcolor=black,citecolor=black]{hyperref}  % 目录可跳转
\setlength{\headheight}{15pt}
\usepackage{geometry}
\usepackage{framed} 
%\geometry{a4paper,scale=0.9}
\geometry{a4paper,left=2cm,right=2cm,top=2cm,bottom=1cm}
% 定义页眉页脚
\pagestyle{fancy}
\fancyhf{}
\fancyhead[C]{ACM Template By Zeng Xiaocan}
\lfoot{}
\cfoot{\thepage}
\rfoot{}

\author{Zeng Xiaocan}   
\title{ACM Template}

\begin{document} 
\maketitle % 封面
\newpage % 换页
\tableofcontents % 目录
\newpage
\section{String}
\subsection{STL}
\begin{framed}
	\noindent reverse(s.begin(), s.end());
	\\ transform(s.begin(), s.end(), s.begin(), ::toupper);  (::tolower)
	\\ //字符串和数字互转
	\\ int a;
	\\ stringstream(s) >> a;
	\\ char s[100];
	\\ sprint(s,"\%d",a);
	\\ string(v.begin(),v.end());
	\\ //返回pos开始的长度为len的字符串
	\\ substr(pos,len);
	\\ //在pos位置插入字符串s
	\\ insert(int pos,string s)
	\\ //从索引pos开始往后删num个,num为空表示全删除
	\\ erase(pos,num);
	\\ //删除迭代器it指向的字符,返回删除后迭代器的位置
	\\ erase(it);
	\\ //删除迭代器[first,last)之间的所有字符,返回删除后迭代器的位置
	\\ erase(first,last);
	\\ //从pos开始查找字符c/字符串s在当前字符串的位置
	\\ int find(c/s,pos);
\end{framed}
\subsection{Max/Min-Expression}
\inputminted[]{c++}{Template/String/express.cpp}
\subsection{KMP}
\inputminted[]{c++}{Template/String/kmp.cpp}
\subsection{EXKMP}
\inputminted[]{c++}{Template/String/exkmp.cpp}
\subsection{Hash}
\inputminted[]{c++}{Template/String/hash.cpp}
\subsection{Trie}
\inputminted[]{c++}{Template/String/trie.cpp}
\subsection{AC-Automaton}
\inputminted[]{c++}{Template/String/acm.cpp}
\subsection{Manacher}
\inputminted[]{c++}{Template/String/manacher.cpp}
\subsection{Palindromic-Tree}
\inputminted[]{c++}{Template/String/pt.cpp}
\subsection{Suffix-Array}
\inputminted[]{c++}{Template/String/sa.cpp}
\inputminted[]{c++}{Template/String/sa-dc3.cpp}
\subsubsection{Usage}
\begin{framed}
	\noindent 0 循环字符串字典序第k小
	\\ 将原串拼接在最后,再加一个大于字符集最大值的字符,计算sa,sa本身就是对后缀进行排序,按顺序枚举k个有效(sa[i]在0-n)的后缀即可。
\end{framed}
\subsection{Suffix-Automaton}
\inputminted[]{c++}{Template/String/sam.cpp}
\subsubsection{Usage}
\begin{framed}
	\noindent 0 判断模式串是否是原串的子串
	\\ 从起点S按模式串的每个字符进行转移,无法转移则不是。
	\\ 1 字符串最小循环移位
	\\ 对字符串s+s建立sam,从起点贪心向最小的字符转移。
	\\ 2 不同子串个数
	\\ -(1)-所有的状态节点就保存了所有不同子串,枚举每个状态,计算$\sum (len[i]-len[fa[i]])$即可。
	\\ 推广到长度大于等于m的不同子串个数,答案即$\sum max(0,len[i]-max(len[fa[i]],m-1))$。
	\\ -(2)-建立sam后直接从根节点(0)dfs搜索,dp[u]表示u为起点的路径数,$dp[u]+=\sum dp[v]$,注意计算过的dp[v]不要重复计算,最后答案是dp[0]-1(或初始化dp[i]为1,dp[0]为0)。
	\\dfs也可以改用拓扑排序,从后往前递推。
	\\ 3 不同字串长度之和
	\\ 即不同路径的长度之和,ans[u]表示u为起点的路径长度和,$ans[u]=\sum (ans[v]+dp[v])$,即(u,v)这条边对每条路径都有一个长度字符的贡献。
	\\ 4 字典序第k小子串(相同子串算1个)
	\\ 从根节点(0)往下走,根据求出的dp[i]和k大小比较,判断走哪一条边,并输出该字符(k也要减1),递归继续判断。
	\\ 5 出现次数k次的不同子串个数。
	\\ 子串出现的次数即endpos的大小,因此求出endpos大小然后枚举所有状态即可。
	\\ 从S开始的反向fa连接可以看成一个parent树,由endpos的性质,$|endpos(u)|=\sum |endpos(v)|+1/0$,是否需要加上1取决于该节点对应的substrings是否包含原串的某个前缀(即非分解出来的状态节点cl)。
	\\ 拓扑(桶?)排序后从后往前推,累加$|endpos|$,节点0代表空串,$|endpos|=0$。
	\\ 6 字典序第k小子串(相同子串算多个)
	\\ 结合上述第4和第5,定义pd[u]表示节点u为起点的子串数(可相同),初始化$pd[i]=|endpos(i)|(i>0)$,而$pd[u]+=\sum pd[v]$。
	\\ 求解的时候,找到满足的字符(pd[v]>=k),直接跳过相同的前缀个数(k-num[u]),递归边界同样是判断(k<=num[u])。
\end{framed}
\subsubsection{Memo}
\inputminted[]{c++}{Template/String/usage/sam-1.cpp}
\subsection{ProblemSet}
%\section{Graph Theory} % 一级标题
%\subsection{Minimum Spanning Tree} % 二级标题
%\subsubsection{Kruskal} % 三级标题
%\inputminted[breaklines]{c++}{graph/kruskal.cc} % 插入代码文件
%% 中文测试
%\subsection{单源最短路}
%\subsubsection{SPFA}
%\inputminted[breaklines]{c++}{graph/spfa.cc}

%\twocolumn  % 分页显示
%\newpage
%\section{String}
%\subsection{KMP}
%\inputminted[breaklines]{c++}{string/kmp.cc}

%\subsection{Suffix Automaton}
%\inputminted[breaklines]{c++}{string/suffix-automaton.cc}

%\newpage
%\section{Others}

\end{document}
