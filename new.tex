\documentclass[a4paper,12pt]{article}
\usepackage{zh_CN-Adobefonts_external} % Simplified Chinese Support using external fonts (./fonts/zh_CN-Adobe/)
\usepackage{fancyhdr}  % 页眉页脚
\usepackage{minted}    % 代码高亮
\usepackage[colorlinks,linkcolor=black,anchorcolor=black,citecolor=black]{hyperref}  % 目录可跳转
\setlength{\headheight}{15pt}
\usepackage{geometry}
\usepackage{framed} 
\geometry{a4paper,scale=0.8}

% 定义页眉页脚
\pagestyle{fancy}
\fancyhf{}
\fancyhead[C]{ACM Template By Zeng Xiaocan}
\lfoot{}
\cfoot{\thepage}
\rfoot{}

\author{Zeng Xiaocan}   
\title{ACM Template}

\begin{document} 
\maketitle % 封面
\newpage % 换页
\tableofcontents % 目录
\newpage
\section{String}
\subsection{STL}
\begin{framed}
	\noindent reverse(s.begin(), s.end());
	\\ transform(s.begin(), s.end(), s.begin(), ::toupper);  (::tolower)
	\\ //字符串和数字互转
	\\ int a;
	\\ stringstream(s) >> a;
	\\ char s[100];
	\\ sprint(s,"\%d",a);
	\\ string(v.begin(),v.end());
	\\ //返回pos开始的长度为len的字符串
	\\ substr(pos,len);
	\\ //在pos位置插入字符串s
	\\ insert(int pos,string s)
	\\ //从索引pos开始往后删num个,num为空表示全删除
	\\ erase(pos,num);
	\\ //删除迭代器it指向的字符,返回删除后迭代器的位置
	\\ erase(it);
	\\ //删除迭代器[first,last)之间的所有字符,返回删除后迭代器的位置
	\\ erase(first,last);
	\\ //从pos开始查找字符c/字符串s在当前字符串的位置
	\\ int find(c/s,pos);
\end{framed}
\subsection{Max/Min-Expression}
\inputminted[]{c++}{Template/String/express.cpp}
\subsection{KMP}
\inputminted[]{c++}{Template/String/kmp.cpp}
\subsection{EXKMP}
\inputminted[]{c++}{Template/String/exkmp.cpp}
\subsection{Hash}
\inputminted[]{c++}{Template/String/hash.cpp}
\subsection{Trie}
\inputminted[]{c++}{Template/String/trie.cpp}
\subsection{AC-Automaton}
\inputminted[]{c++}{Template/String/acm.cpp}
\subsection{Manacher}
\inputminted[]{c++}{Template/String/manacher.cpp}
\subsection{Palindromic-Tree}
\inputminted[]{c++}{Template/String/pt.cpp}
\subsection{Suffix-Array}
%\inputminted[]{c++}{Template/String/sa.cpp}
%\inputminted[]{c++}{Template/String/sa-dc3.cpp}
\subsection{Suffix-Automaton}
\inputminted[]{c++}{Template/String/sam.cpp}
\subsection{ProblemSet}
%\section{Graph Theory} % 一级标题
%\subsection{Minimum Spanning Tree} % 二级标题
%\subsubsection{Kruskal} % 三级标题
%\inputminted[breaklines]{c++}{graph/kruskal.cc} % 插入代码文件
%% 中文测试
%\subsection{单源最短路}
%\subsubsection{SPFA}
%\inputminted[breaklines]{c++}{graph/spfa.cc}

%\twocolumn  % 分页显示
%\newpage
%\section{String}
%\subsection{KMP}
%\inputminted[breaklines]{c++}{string/kmp.cc}

%\subsection{Suffix Automaton}
%\inputminted[breaklines]{c++}{string/suffix-automaton.cc}

%\newpage
%\section{Others}

\end{document}
