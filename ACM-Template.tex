\documentclass[a4paper,11pt]{article}
\usepackage{zh_CN-Adobefonts_external} % Simplified Chinese Support using external fonts (./fonts/zh_CN-Adobe/)
\usepackage{fancyhdr}  % 页眉页脚
\usepackage{minted}    % 代码高亮
\usepackage[colorlinks,linkcolor=black,anchorcolor=black,citecolor=black]{hyperref}  % 目录可跳转
\setlength{\headheight}{15pt}
\usepackage{geometry}
\geometry{a4paper,scale=0.8}

% 定义页眉页脚
\pagestyle{fancy}
\fancyhf{}
\fancyhead[C]{ACM Template By Zeng Xiaocan}
\lfoot{}
\cfoot{\thepage}
\rfoot{}

\author{Zeng Xiaocan}   
\title{ACM Template}

\begin{document} 
\maketitle % 封面
\newpage % 换页
\tableofcontents % 目录
\newpage
\section{字符串}
\subsection{KMP}
\inputminted[]{c++}{Template/String/KMP.cpp}
\subsection{Trie树/AC自动机}
\inputminted[]{c++}{Template/String/TrieAC.cpp}
\subsection{Manacher}
\inputminted[]{c++}{Template/String/Manacher.cpp}
\subsection{后缀数组}
\inputminted[]{c++}{Template/String/Suffix-Array.cpp}
% \subsection{后缀数组经典应用}
% \inputminted[]{c++}{Template/String/SA-usage.cpp}
\subsection{字符串哈希}
\inputminted[]{c++}{Template/String/Hash.cpp}
\subsection{其他}
\subsubsection{最大/小表示法}
\inputminted[]{c++}{Template/String/Express.cpp}
\section{树图}
\subsection{拓扑排序}
\inputminted[]{c++}{Template/TreeGraph/TopoSort.cpp}
\subsection{并查集}
\subsubsection{普通并查集}
\inputminted[]{c++}{Template/TreeGraph/UnionSetI.cpp}
\subsubsection{带权并查集}
\inputminted[]{c++}{Template/TreeGraph/UnionSetII.cpp}
\subsection{最小生成树}
\subsubsection{Prim}
\inputminted[]{c++}{Template/TreeGraph/MST-Prim.cpp}
\subsection{次小生成树}
\inputminted[]{c++}{Template/TreeGraph/SST.cpp}
\subsection{tarjan}
\subsubsection{强连通分量}
\inputminted[]{c++}{Template/TreeGraph/Tarjan-SCC.cpp}
\subsubsection{边-双连通分量}
\inputminted[]{c++}{Template/TreeGraph/Tarjan-Edge-BCC.cpp}
\subsubsection{点-双连通分量}
\inputminted[]{c++}{Template/TreeGraph/Tarjan-Vertex-BCC.cpp}
\subsection{LCA}
\subsubsection{ST表在线}
\inputminted[]{c++}{Template/TreeGraph/LCA-ST.cpp}
\subsubsection{Tarjan离线}
\inputminted[]{c++}{Template/TreeGraph/LCA-Tarjan.cpp}
\subsection{最短路}

\subsection{网络流}
\subsubsection{Dinic}
\inputminted[]{c++}{Template/TreeGraph/Dinic.cpp}
\subsubsection{上下界网络流}
\inputminted[]{c++}{Template/TreeGraph/UpDownFlow.cpp}
\section{动态规划}
\subsection{子序列/子串}
\subsubsection{最大连续子序列和}
\inputminted[]{c++}{Template/DynamicProgramming/MaxSubSum.cpp}
\subsubsection{最大上升/下降子序列}
\inputminted[]{c++}{Template/DynamicProgramming/LIS.cpp}
\subsection{背包}
\subsection{数位dp}
\inputminted[]{c++}{Template/DynamicProgramming/Digital-dp.cpp}
\subsection{区间dp}
\subsubsection{石头合并问题}
\inputminted[]{c++}{Template/DynamicProgramming/Inteval-dp.cpp}
\subsection{其他}
\subsubsection{编辑距离}
\inputminted[]{c++}{Template/DynamicProgramming/LevenshteinDistance.cpp}
\section{基础数论}
\subsection{快速幂取模}
\inputminted[]{c++}{Template/Math/PowMod.cpp}
\subsection{欧拉函数}
\inputminted[]{c++}{Template/Math/Euler.cpp}
\subsection{欧拉降幂}
\inputminted[]{c++}{Template/Math/EulerPower.cpp}
\subsection{拓展欧几里得(EXGCD)}
\inputminted[]{c++}{Template/Math/exgcd.cpp}
\subsection{中国剩余定理(CRT)}

\subsection{逆元}
\inputminted[]{c++}{Template/Math/Inverse.cpp}
\subsection{小技巧}
\subsubsection{求n!位数}
\inputminted[]{c++}{Template/Math/CountN!.cpp}
\section{博弈}
\subsection{SG函数}
\inputminted[]{c++}{Template/Game/SG.cpp}
\subsection{Bash Game}
\inputminted[]{c++}{Template/Game/Bash.cpp}
\subsection{Wythoff Game}
\inputminted[]{c++}{Template/Game/Wythoff.cpp}
\subsection{Nim Game}
\inputminted[]{c++}{Template/Game/Nim.cpp}
\subsection{Fibonaci Game}
\inputminted[]{c++}{Template/Game/Fibonaci.cpp}
\section{计算几何}

\section{区间问题}
\subsection{线段树}

\subsection{RMQ}
\inputminted[]{c++}{Template/Segment/RMQ.cpp}
\subsection{树状数组}
\subsubsection{单点更新区间求和}
\inputminted[]{c++}{Template/Segment/TreeArrayI.cpp}
\subsubsection{求逆序数}
\inputminted[]{c++}{Template/Segment/Reverse.cpp}
\subsubsection{区间更新单点查询}
\inputminted[]{c++}{Template/Segment/TreeArrayII.cpp}
\subsubsection{区间更新区间求和}
\inputminted[]{c++}{Template/Segment/TreeArrayIII.cpp}
\subsubsection{二维树状数组--单点更新区间求和}
\inputminted[]{c++}{Template/Segment/TwoDimTreeArrayI.cpp}
\subsubsection{二维树状数组--区间更新单点查询}
\inputminted[]{c++}{Template/Segment/TwoDimTreeArrayII.cpp}
\section{其他}
\subsection{双指针/尺取法}
\subsubsection{一维}
\inputminted[]{c++}{Template/Other/TwoPointer.cpp}
\subsubsection{二维}
\inputminted[]{c++}{Template/Other/TwoDimTwoPointer.cpp}
\subsection{单调队列/单调栈}
\subsubsection{最大m子段和}
\inputminted[]{c++}{Template/Other/MonotonicQueueI.cpp}
\subsubsection{m区间最小值}
\inputminted[]{c++}{Template/Other/MonotonicQueueII.cpp}
\subsection{矩阵快速幂}
\inputminted[]{c++}{Template/Other/FastMat.cpp}
\subsection{判断重边}
\inputminted[]{c++}{Template/Other/Edge-Judge-Repeat.cpp}
\subsection{BM递推}
\inputminted[]{c++}{Template/Other/BM.cpp}
\subsection{大数平方数判断}
\inputminted[]{java}{Template/Other/IsSquare.java}
\subsection{技巧}
\subsubsection{快读}
\subsubsection{离散化}
\inputminted[]{c++}{Template/Other/Discretization.cpp}
\subsection{一些比较有意思的题目}
\subsubsection{hdu6468——求1-n字典序第m个数}
\inputminted[]{c++}{Template/Other/hdu6468.cpp}
%\section{Graph Theory} % 一级标题
%\subsection{Minimum Spanning Tree} % 二级标题
%\subsubsection{Kruskal} % 三级标题
%\inputminted[breaklines]{c++}{graph/kruskal.cc} % 插入代码文件
%% 中文测试
%\subsection{单源最短路}
%\subsubsection{SPFA}
%\inputminted[breaklines]{c++}{graph/spfa.cc}

%\twocolumn  % 分页显示
%\newpage
%\section{String}
%\subsection{KMP}
%\inputminted[breaklines]{c++}{string/kmp.cc}

%\subsection{Suffix Automaton}
%\inputminted[breaklines]{c++}{string/suffix-automaton.cc}

%\newpage
%\section{Others}

\end{document}
