\documentclass[a4paper,11pt]{article}
\usepackage{zh_CN-Adobefonts_external} % Simplified Chinese Support using external fonts (./fonts/zh_CN-Adobe/)
\usepackage{fancyhdr}  % 页眉页脚
\usepackage{minted}    % 代码高亮
\setlength{\headheight}{15pt}
\usepackage{geometry}
\geometry{a4paper,scale=0.8}

% 定义页眉页脚
\pagestyle{fancy}
\fancyhf{}
\fancyhead[C]{ACM Template By Zeng Xiaocan}
%\lfoot{}
%\cfoot{\thepage}
%\rfoot{}

%\author{Zeng Xiaocan}   
%\title{ACM Template}

\begin{document}\small
%\section*{multiset}
% 分模块 要更新模板只需要把每个模块最后一页重新拼接打印即可

% 字符串
\newpage
\section*{string}
\inputminted[]{c++}{Template/String/STL.cpp}
\section*{KMP}
\inputminted[]{c++}{Template/String/KMP.cpp}
\section*{Manacher}
\inputminted[]{c++}{Template/String/Manacher.cpp}
\section*{AC自动机}
\inputminted[]{c++}{Template/String/TrieAC.cpp}
\section*{后缀数组}
\subsection*{倍增}
\inputminted[]{c++}{Template/String/Suffix-Array.cpp}
\subsection*{DC3}
\inputminted[]{c++}{Template/String/Suffix-Array-DC3.cpp}
\subsection*{后缀数组经典应用}
\inputminted[]{c++}{Template/String/SA-usage.cpp}
\section*{字符串哈希}
\inputminted[]{c++}{Template/String/Hash.cpp}
\section*{最大/小表示法}
\inputminted[]{c++}{Template/String/Express.cpp}

% 树图
\newpage
\section*{普通并查集}
\inputminted[]{c++}{Template/TreeGraph/UnionSetI.cpp}
\section*{带权并查集}
\inputminted[]{c++}{Template/TreeGraph/UnionSetII.cpp}
\section*{Dijkstra堆优化}
\inputminted[]{c++}{Template/TreeGraph/Dijkstra.cpp}
\section*{次小生成树}
\inputminted[]{c++}{Template/TreeGraph/SST.cpp}
\section*{拓扑排序}
\inputminted[]{c++}{Template/TreeGraph/TopoSort.cpp}
\section*{最小支配集/最小点覆盖/最大独立集}
\inputminted[]{c++}{Template/TreeGraph/SetPro.cpp}
\section*{树的直径/重心}
\inputminted[]{c++}{Template/TreeGraph/TreeDiameter.cpp}
%\section*{树的点分治}
\subsection*{强连通分量}
\inputminted[]{c++}{Template/TreeGraph/Tarjan-SCC.cpp}
\section*{边-双连通分量}
\inputminted[]{c++}{Template/TreeGraph/Tarjan-Edge-BCC.cpp}
\section*{点-双连通分量}
\inputminted[]{c++}{Template/TreeGraph/Tarjan-Vertex-BCC.cpp}
\section*{LCA}
\subsection*{ST表在线}
\inputminted[]{c++}{Template/TreeGraph/LCA-ST.cpp}
\subsection*{Tarjan离线}
\inputminted[]{c++}{Template/TreeGraph/LCA-Tarjan.cpp}
\subsection*{倍增在线}
\inputminted[]{c++}{Template/TreeGraph/LCA-Mul.cpp}

% 二分图/网络流
\newpage
\section*{判断二分图}
\inputminted[]{c++}{Template/NetworkFlow/BinaryJudge.cpp}
\section*{最大匹配}
\inputminted[]{c++}{Template/NetworkFlow/MaximumMatch.cpp}
\section*{完美匹配}
\inputminted[]{c++}{Template/NetworkFlow/PerfectMatch.cpp}
\section*{最优匹配(KM算法)}
\inputminted[]{c++}{Template/NetworkFlow/KM.cpp}
\section*{最小支配集/最小点覆盖/最大独立集/最大团/最小路径覆盖}
\inputminted[]{c++}{Template/NetworkFlow/SetPro.cpp}
\section*{多重匹配}
\inputminted[]{c++}{Template/NetworkFlow/MultiMatch.cpp}
\section*{最大流(Dinic)}
\inputminted[]{c++}{Template/NetworkFlow/Dinic.cpp}
\section*{最小费用最大流}
\inputminted[]{c++}{Template/NetworkFlow/mcmf.cpp}
\section*{上下界网络流}
\inputminted[]{c++}{Template/NetworkFlow/UpDownFlow.cpp}
\section*{常见模型}
\inputminted[]{c++}{Template/NetworkFlow/FlowModel.cpp}

% 区间问题
\newpage
\section*{线段树}
\subsection*{单点更新区间查询}
\inputminted[]{c++}{Template/Segment/SegTreeI.cpp}
\subsection*{区间更新}
\inputminted[]{c++}{Template/Segment/SegTreeII.cpp}
\subsection*{多标记}
\inputminted[]{c++}{Template/Segment/SegTreeIII.cpp}
\section*{权值线段树}
\inputminted[]{c++}{Template/Segment/WeightSegTree.cpp}
\section*{扫描线}
\subsection*{矩形周长并}
\inputminted[]{c++}{Template/Segment/poj1177.cpp}
\subsection*{矩形面积并}
\inputminted[]{c++}{Template/Segment/hdu1542.cpp}
\subsection*{矩形面积交}
\inputminted[]{c++}{Template/Segment/hdu1255.cpp}
\subsection*{立方体体积交}
\inputminted[]{c++}{Template/Segment/hdu3642.cpp}
\section*{区间合并}
\inputminted[]{c++}{Template/Segment/SegmentMerge.cpp}
\section*{其他常见模型}
\inputminted[]{c++}{Template/Segment/OtherModel.cpp}
\section*{RMQ}
\inputminted[]{c++}{Template/Segment/RMQ.cpp}
\section*{树状数组}
\subsection*{单点更新区间求和}
\inputminted[]{c++}{Template/Segment/TreeArrayI.cpp}
\subsection*{求逆序数}
\inputminted[]{c++}{Template/Segment/Reverse.cpp}
\subsection*{区间更新单点查询}
\inputminted[]{c++}{Template/Segment/TreeArrayII.cpp}
\subsection*{区间更新区间求和}
\inputminted[]{c++}{Template/Segment/TreeArrayIII.cpp}
\section*{二维树状数组}
\subsection*{单点更新区间求和}
\inputminted[]{c++}{Template/Segment/TwoDimTreeArrayI.cpp}
\subsection*{区间更新单点查询}
\inputminted[]{c++}{Template/Segment/TwoDimTreeArrayII.cpp}
\section*{莫队}
\subsection*{普通莫队}
\inputminted[]{c++}{Template/Segment/bzoj2038.cpp}
\subsection*{带修莫队}
\inputminted[]{c++}{Template/Segment/bzoj2120.cpp}
\subsection*{树上莫队}
\inputminted[]{c++}{Template/Segment/luoguP4074.cpp}
%\section*{主席树几种应用}
\section*{可持久化字典树}
\subsection*{树上路径异或最大值}
\inputminted[]{c++}{Template/Segment/PersistibleTrieI.cpp}
\subsection*{子树异或最大值}
\inputminted[]{c++}{Template/Segment/PersistibleTrieII.cpp}
%\subsection*{带修改序列异或最大值}

% 其他
\newpage
\section*{双指针}
\subsection*{一维}
\inputminted[]{c++}{Template/Other/TwoPointer.cpp}
\subsection*{二维}
\inputminted[]{c++}{Template/Other/TwoDimTwoPointer.cpp}
\section*{单调队列/单调栈}
\subsection*{最大m子段和}
\inputminted[]{c++}{Template/Other/MonotonicQueueI.cpp}
\subsection*{m区间最小值}
\inputminted[]{c++}{Template/Other/MonotonicQueueII.cpp}
\subsection*{作为最大/最小值能延伸的区间}
\inputminted[]{c++}{Template/Other/MonotonicQueueIII.cpp}
\section*{矩阵快速幂}
\inputminted[]{c++}{Template/Other/FastMat.cpp}
\section*{BM递推}
\inputminted[]{c++}{Template/Other/BM.cpp}
\section*{大数平方数判断}
\inputminted[]{java}{Template/Other/IsSquare.java}
\section*{快读}
\subsection*{•}*{c++}
\inputminted[]{c++}{Template/Other/fread.cpp}
\subsection*{java}
\inputminted[]{java}{Template/Other/fScan.java}
\section*{离散化}
\inputminted[]{c++}{Template/Other/Discretization.cpp}
\section*{mt19937随机数}
\inputminted[]{c++}{Template/Other/mt19937.cpp}
\section*{bitset}
\inputminted[]{c++}{Template/Other/bitset.cpp}

\newpage
\section*{hdu6468——求1-n字典序第m个数}
\inputminted[]{c++}{Template/Other/hdu6468.cpp}
\section*{cf1144E——求字符串中位数(26进制模拟)}
\inputminted[]{c++}{Template/Other/cf1144E.cpp}
\section*{牛客548B——除法模拟}
\inputminted[]{c++}{Template/Other/548B.cpp}
\section*{hdu4507——数位dp求平方和}
\inputminted[]{c++}{Template/Other/hdu4507.cpp}

\end{document}
